\begin{definition}
\begin{align}
    \sbrak{f (t)}=\mathcal{L}\cbrak{u (t) f (t)}
\end{align}
\end{definition}
\begin{definition}[Laplace Transform]
    It is an integral transform that converts a function of a real variable $t$ to a function of a complex variable $s$. The Laplace transform of $f(t)$ is denoted by $\mathcal{L}\cbrak{f(t)}$ or $F(s)$.
    \begin{align}
        F(s)=\mathcal{L}\cbrak{f(t)}=\int_{0}^{\infty}e^{-st}f(t)dt
    \end{align}
\end{definition}
\begin{lemma}\label{ec/1999/1/3/lemma:timeShift}
    Time Shift Property of Laplace transformation
    \begin{align}
        f(t-T)u(t-T) \laplace e^{-sT}F(s)
    \end{align}
\end{lemma}
\begin{proof}
    \begin{multline}
        \mathcal{L}\cbrak{f(t-T)u(t-T)}\\=\int_{0}^{\infty}e^{-st}f(t-T)u(t-T)dt\\
    \end{multline}
    This can be written as
    \begin{align}
        &=e^{-sT}\int_{-T}^{\infty}e^{-s(t-T)}f(t-T)u(t-T)d(t-T)\\
        &=e^{-sT}\int_{0}^{\infty}e^{-s(t-T)}f(t-T)d(t-T)\\
        &=e^{-sT}F(s)
    \end{align}

\end{proof}

Using~\eqref{ec/1999/1/3/lemma:timeShift}, the correct answer is \textbf{Option(2)}
