From the given information, 
\begin{equation}
    \myvec{\dot{x}(t)\\y(t)} = \myvec{A & B\\C & d}\myvec{x(t)\\u(t)}
\end{equation}
%    
Taking Laplace transform on both sides,
\begin{align}
    \myvec{sX(s)\\Y(s)} &= \myvec{A & B\\ C & d}\myvec{X(s)\\U(s)}\\
    \implies sX(s) &= AX(s)+BU(s)\\
    \implies X(s) &= (sI-A)^{-1} BU(s)\\
    \implies Y(s) &= CX(s)+dU(s)\\ 
                  &= C(sI-A)^{-1} BU(s) +dU(s)
\end{align}

By definition, 
\begin{align}
    Y(s) &= H(s)U(s)\\
    \implies H(s) &= C(sI-A)^{-1} B + d\\
                  &= \dfrac{1}{s^3+3s^2+2s+1}
\end{align} 
\begin{equation}
    \implies C(sI-A)^{-1} B + d = \dfrac{1}{s^3+3s^2+2s+1}\label{ec/2019/33result}
\end{equation}

Now we cross verify the options with eq \ref{ec/2019/33result}. By using a python script, 
\begin{enumerate}[label = (\Alph*)]
    \item \begin{equation}
        C(sI-A)^{-1} B +d = \dfrac{1}{s^3+3s^2+2s+1}
    \end{equation}
    \item \begin{equation}
        C(sI-A)^{-1} B +d = \dfrac{1}{s^3+1s^2+2s+3}
    \end{equation}    
    \item \begin{equation}
        C(sI-A)^{-1} B +d = \dfrac{s^2}{s^3+3s^2+2s+1}
    \end{equation}
    \item \begin{equation}
        C(sI-A)^{-1} B +d = \dfrac{s^2}{s^3+1s^2+2s+3}
    \end{equation}
\end{enumerate}

Hence  A is the correct option.

